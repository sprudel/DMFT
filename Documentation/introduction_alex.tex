
\section{Introduction and Overview}
The aim of this project is to study the metal to Mott-Insulator phase transition exhibited by the Fermi-Hubbard model.

The model consists of a lattice with a single-level \textit{atom} at every site. The electrons can only hop from a site to a nearest neighbour one, and only interact between them if they're at the same site. The Hamiltonian of this model is, therefore, given by:

\begin{equation}
\mathcal{H} = -t\sum_{<i,j>}c_{i,\sigma}^\dagger c_{j,\sigma} + h.c.+ U\sum_i n_{i,\uparrow}n_{i,\downarrow} + \mu\sum_i (n_{i,\uparrow}n_{i,\downarrow})
\end{equation}
where $t$ is the hopping rate, $U$ the strenght of the interaction, varying which we will observe the P.T., and $\mu$ the chemical potential.




This model is studied here by means of Dynamical Mean Field Theory~(DMFT), and the quantity of interest is the local Green's function, given by:

\begin{equation}\label{orGreen's}
G_{ii}^\sigma (\tau - \tau^\prime) = -\braket{Tc_{i\sigma}(\tau)c_{i\sigma}^\dagger(\tau^\prime)} 
\end{equation}
by means of which it will be then possible to compute the \emph{Spectral Function}.

~

The main idea behind the DMFT approach is similar in spirit to the classical mean field approximation and consists in solving the problem of a single atom coupled to a thermal bath and mapping this to our original lattice problem via a self-consistency relation.
Such single atom problem is described by the Hamiltonian of a so called \emph{Anderson Impurity Model} (AIM), given by:

\begin{equation}
\mathcal{H}_{AIM} = \mathcal{H}_{atom} + \mathcal{H}_{bath} + \mathcal{H}_{coupling}
\end{equation}
where we have the following:
\begin{equation}\label{AIMH}
\begin{array}{c}
\mathcal{H}_{atom} = Un_\uparrow^cn_\downarrow^c + (\epsilon_0 - \mu)(n_\uparrow^c+n_\downarrow^c) \\ \\ \mathcal{H}_{bath} = \sum_{l,\sigma}\tilde{\epsilon}_la_{l\sigma}^\dagger a_{l\sigma} \\ \\ \mathcal{H}_{coupling} = \sum_{l,\sigma}V_l(a_{l\sigma}^\dagger c_{\sigma} + c_{\sigma}^\dagger a_{l\sigma})
\end{array}
\end{equation}
here, the the $a_l$'s describe the fermionic degrees of freedom of the bath, while the $\tilde{\epsilon_l}'s$ and the $V_l$'s are parameters which must been chosen appopriately (such that the impurity Green's funtion of \eqref{AIMH} coincides with the local lattice one) and enter through the hybridisation function:

\begin{equation}
\Delta(i\omega_n) = \sum_l\frac{|V_l|^2}{i\omega_n - \tilde{\epsilon}_l}
\end{equation}
how it can be seen from the effective action for the system, obtained integrating out the bath degrees of freedom:
\begin{equation}
S_{eff} = -\int_0^\beta\int_0^\beta d\tau d\tau^\prime\sum_\sigma c_\sigma^\dagger(\tau)\mathscr{G}_0^{-1}(\tau-\tau^\prime)c_\sigma(\tau^\prime) + U\int_0^\beta d\tau n_\uparrow(\tau)n_\downarrow(\tau)
\end{equation}
where we have defined:
\begin{equation}
\mathscr{G}_0^{-1}(i\omega_n) = i\omega_n + \mu - \epsilon_0 - \Delta(i\omega_n) 
\end{equation}

At this point come into play the mean field approximation. First of all, we notice that we can define a local self-energy for the interacting Green's function of the effective AIM, $G(\tau-\tau^\prime)$, via:

\begin{equation}\label{s_imp}
\Sigma_{imp}(i\omega_n) \equiv \mathscr{G}_0^{-1}(i\omega_n) - G^{-1}(i\omega_n)
\end{equation}
And, of course, we can also consider the self-energy of our origina lattice problem, defined from \eqref{orGreen's}, via:
\begin{equation}\label{not_summed_over}
G(\bold{k},i\omega_n) = \frac{1}{i\omega_n + \mu - \epsilon_0 - \epsilon_{\bold{k}} -\Sigma(\bold{k},i\omega_n)}
\end{equation}
with:
\begin{equation}
\epsilon_\bold{k} \equiv t\sum_je^{i\bold{k}\cdot(\bold{R_i}-\bold{R_j})}
\end{equation}
The approximation, now, consists of saying that the lattice self-energy coincides with the impurity self-energy, resulting in vanishing off-diagonal elements of $\Sigma_{latt}$:

\begin{equation}
\Sigma_{ii} \simeq \Sigma_{imp} ~ , \Sigma_{i\neq j} \simeq 0
\end{equation}
which is a consistent approximation only given that it uniquely determines the local Green's function, which, by assumption, is the impurity problem Green's function. We, therefore, sum \eqref{not_summed_over} over $\bold{k}$ to obtain \eqref{orGreen's}, and use \eqref{s_imp} to arrive to the self-consistency relation:
\begin{equation}
\sum_\bold{k}\frac{1}{\Delta(i\omega_n)+G(i\omega_n)^{-1}-\epsilon_\bold{k}} = G(i\omega_n)
\end{equation}

This is the idea behind the DMFT approach. In practice one use an iterative procedure, following the loop: 
\begin{enumerate}
\item start with an initial guess for $\mathscr{G}_0$ (i.e. for $\Delta$);
\item compute the AIM Green's function $G_{imp}$ (by means of perturbation theory, in our case up to second order) $\rightarrow$ $\Sigma_{imp}$ is computed;
\item compute the lattice problem local Green's function $G_{loc}$;
\item update $\mathscr{G}_0$ via $\mathscr{G}_{0,new}^{-1} = G_{loc}^{-1} + \Sigma_{imp}$;
\item iterate till convergence.
\end{enumerate}
which is what we have done in the project. Finally, once the lattice local Green's function has been obtained for the set of values $\{i\omega_n\}$, we fit it using the Padé approximation, and, eventually, we are able to compute the Spectral Function, via analytic continuation of this fit.


