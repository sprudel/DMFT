\documentclass[11pt]{article}
\usepackage{geometry}
\geometry{a4paper, top=27mm, left=25mm, right=20mm, bottom=20mm, headsep=0mm, footskip=12mm}

\usepackage[english]{babel}
\usepackage[utf8]{inputenc}
\usepackage{uniinput}
\usepackage{amsmath,amsfonts,amssymb}
\usepackage{commath}
\usepackage{braket}
\usepackage{mathrsfs}
\usepackage{graphicx}
\usepackage{todonotes}
\title{DMFT with Iterated Perturbation Theory}
\author{Fabian Kugler, Hannes Herrmann and Alessandro Bottero}

\begin{document}
\maketitle

\section{Introduction and Overview}
The aim of this project is to study the metal to Mott-Insulator phase transition exhibited by the Fermi-Hubbard model.

The model consists of a lattice with a single-level \textit{atom} at every site. The electrons can only hop from a site to a nearest neighbour one, and only interact between them if they're at the same site. The Hamiltonian of this model is, therefore, given by:

\begin{equation}
\mathcal{H} = -t\sum_{<i,j>}c_{i,\sigma}^\dagger c_{j,\sigma} + h.c.+ U\sum_i n_{i,\uparrow}n_{i,\downarrow} + \mu\sum_i (n_{i,\uparrow}n_{i,\downarrow})
\end{equation}
where $t$ is the hopping rate, $U$ the strenght of the interaction, varying which we will observe the P.T., and $\mu$ the chemical potential.




This model is studied here by means of Dynamical Mean Field Theory~(DMFT), and the quantity of interest is the local Green's function, given by:

\begin{equation}\label{orGreen's}
G_{ii}^\sigma (\tau - \tau^\prime) = -\braket{Tc_{i\sigma}(\tau)c_{i\sigma}^\dagger(\tau^\prime)} 
\end{equation}
by means of which it will be then possible to compute the \emph{Spectral Function}.

~

The main idea behind the DMFT approach is similar in spirit to the classical mean field approximation and consists in solving the problem of a single atom coupled to a thermal bath and mapping this to our original lattice problem via a self-consistency relation.
Such single atom problem is described by the Hamiltonian of a so called \emph{Anderson Impurity Model} (AIM), given by:

\begin{equation}
\mathcal{H}_{AIM} = \mathcal{H}_{atom} + \mathcal{H}_{bath} + \mathcal{H}_{coupling}
\end{equation}
where we have the following:
\begin{equation}\label{AIMH}
\begin{array}{c}
\mathcal{H}_{atom} = Un_\uparrow^cn_\downarrow^c + (\epsilon_0 - \mu)(n_\uparrow^c+n_\downarrow^c) \\ \\ \mathcal{H}_{bath} = \sum_{l,\sigma}\tilde{\epsilon}_la_{l\sigma}^\dagger a_{l\sigma} \\ \\ \mathcal{H}_{coupling} = \sum_{l,\sigma}V_l(a_{l\sigma}^\dagger c_{\sigma} + c_{\sigma}^\dagger a_{l\sigma})
\end{array}
\end{equation}
here, the the $a_l$'s describe the fermionic degrees of freedom of the bath, while the $\tilde{\epsilon_l}'s$ and the $V_l$'s are parameters which must been chosen appopriately (such that the impurity Green's funtion of \eqref{AIMH} coincides with the local lattice one) and enter through the hybridisation function:

\begin{equation}
\Delta(i\omega_n) = \sum_l\frac{|V_l|^2}{i\omega_n - \tilde{\epsilon}_l}
\end{equation}
how it can be seen from the effective action for the system, obtained integrating out the bath degrees of freedom:
\begin{equation}
S_{eff} = -\int_0^\beta\int_0^\beta d\tau d\tau^\prime\sum_\sigma c_\sigma^\dagger(\tau)\mathscr{G}_0^{-1}(\tau-\tau^\prime)c_\sigma(\tau^\prime) + U\int_0^\beta d\tau n_\uparrow(\tau)n_\downarrow(\tau)
\end{equation}
where we have defined:
\begin{equation}
\mathscr{G}_0^{-1}(i\omega_n) = i\omega_n + \mu - \epsilon_0 - \Delta(i\omega_n) 
\end{equation}

At this point come into play the mean field approximation. First of all, we notice that we can define a local self-energy for the interacting Green's function of the effective AIM, $G(\tau-\tau^\prime)$, via:

\begin{equation}\label{s_imp}
\Sigma_{imp}(i\omega_n) \equiv \mathscr{G}_0^{-1}(i\omega_n) - G^{-1}(i\omega_n)
\end{equation}
And, of course, we can also consider the self-energy of our origina lattice problem, defined from \eqref{orGreen's}, via:
\begin{equation}\label{not_summed_over}
G(\bold{k},i\omega_n) = \frac{1}{i\omega_n + \mu - \epsilon_0 - \epsilon_{\bold{k}} -\Sigma(\bold{k},i\omega_n)}
\end{equation}
with:
\begin{equation}
\epsilon_\bold{k} \equiv t\sum_je^{i\bold{k}\cdot(\bold{R_i}-\bold{R_j})}
\end{equation}
The approximation, now, consists of saying that the lattice self-energy coincides with the impurity self-energy, resulting in vanishing off-diagonal elements of $\Sigma_{latt}$:

\begin{equation}
\Sigma_{ii} \simeq \Sigma_{imp} ~ , \Sigma_{i\neq j} \simeq 0
\end{equation}
which is a consistent approximation only given that it uniquely determines the local Green's function, which, by assumption, is the impurity problem Green's function. We, therefore, sum \eqref{not_summed_over} over $\bold{k}$ to obtain \eqref{orGreen's}, and use \eqref{s_imp} to arrive to the self-consistency relation:
\begin{equation}
\sum_\bold{k}\frac{1}{\Delta(i\omega_n)+G(i\omega_n)^{-1}-\epsilon_\bold{k}} = G(i\omega_n)
\end{equation}

This is the idea behind the DMFT approach. In practice one use an iterative procedure, following the loop: 
\begin{enumerate}
\item start with an initial guess for $\mathscr{G}_0$ (i.e. for $\Delta$);
\item compute the AIM Green's function $G_{imp}$ (by means of perturbation theory, in our case up to second order) $\rightarrow$ $\Sigma_{imp}$ is computed;
\item compute the lattice problem local Green's function $G_{loc}$;
\item update $\mathscr{G}_0$ via $\mathscr{G}_{0,new}^{-1} = G_{loc}^{-1} + \Sigma_{imp}$;
\item iterate till convergence.
\end{enumerate}
which is what we have done in the project. Finally, once the lattice local Green's function has been obtained for the set of values $\{i\omega_n\}$, we fit it using the Padé approximation, and, eventually, we are able to compute the Spectral Function, via analytic continuation of this fit.




\section{The impurity problem in 2\textsuperscript{nd} order perturbation theory}

Translating the Hamiltonian formalism into a functional integral one, we get the action
%
\begin{gather*}
S = \
	%int \sum_{\sigma} \bar{c}_{\sigma} (\tau) \partial_{\tau} c_{\sigma} (\tau) 
	%+ H_{\text{atom}} ( \bar{c}_{\sigma} (\tau), c_{\sigma} (\tau))
	%+ \sum_{l, \sigma} \bar{a}_{\sigma} (\tau) \partial_{\tau} a_{\sigma} (\tau) \\
	%+ H_{\text{\text{bath}}} ( \bar{a}_{\sigma} (\tau), a_{\sigma} (\tau)) 
	%+ H_{\text{\text{coupling}}} (\bar{c}_{\sigma} (\tau), c_{\sigma} (\tau), \bar{a}_{\sigma} (\tau), a_{\sigma} (\tau))
	%\dif \tau
	\int_0^{\beta} \sum_{\sigma} \bar{c}_{\sigma} (\tau) \partial_{\tau} c_{\sigma} (\tau) 
	+ \sum_{l, \sigma} \bar{a}_{\sigma} (\tau) \partial_{\tau} a_{\sigma} (\tau)
	+ H_{\text{AIM}} \big( \bar{c}_{\sigma} (\tau), c_{\sigma} (\tau), \bar{a}_{\sigma} (\tau), a_{\sigma} (\tau) \big)
	\dif \tau
	\\
	 = \int_0^{\beta}  H_{\text{atom}} \big( \bar{c}_{\sigma} (\tau), c_{\sigma} (\tau) \big)  \dif \tau 
	+ \sum_{\sigma, \omega} \bar{c}_{\sigma, \omega} 
	\Big( \sum_{l} \frac{V_l}{i\omega - \tilde{\epsilon}_l} - i\omega \Big)
	c_{\sigma, \omega} 
	\\
	+ \sum_{l, \sigma, \omega} \big( \bar{a}_{l, \sigma, \omega} + \frac{V_l}{\tilde{\epsilon}_l - i\omega} \bar{c}_{\sigma, \omega} \big) 
	\big( \tilde{\epsilon_l} - i\omega \big)
	\big( a_{l, \sigma, \omega} + \frac{V_l}{\tilde{\epsilon}_l - i\omega} c_{\sigma, \omega} \big) 
\end{gather*}
%
where some arrangements and usage of the usual Matsubara-Fourier transform was made. We use the convention
$
c_{\sigma} (\tau) = \sum_{\omega} e^{-i\omega \tau} c_{\sigma, \omega}
$
where the sum runs over fermionic Matsubara frequencies and a prefactor of $1/ \beta$ is understood, such that $c_{\sigma, \omega}$ has the dimension of inverse energy. Correspondingly, a Kronecker-delta of Matsubara frequencies contains a factor of $\beta$. In the above expression, the bath can easily be integrated out. The case of half filling, $\mu = U/2$, can be equivalently written with a modified interaction and zero chemical potential. Dropping a constant energy term, one has
%
\begin{gather*}
S_{\text{eff}} = S_0 + S_{\text{int}} =
	 - \sum_{\sigma, \omega} \bar{c}_{\sigma, \omega} G_{0, \omega}^{-1}
	c_{\sigma, \omega} 
	%\\
	+ U \sum_{Q} \Big( 
	\underbrace{
	\sum_{k} \bar{c}_{\uparrow,k+Q} c_{\uparrow,k} - \frac{1}{2} \delta_{Q,0} 
	}_{C_Q}
	\Big) \Big(
	\underbrace{
	\sum_{q} \bar{c}_{\downarrow,q-Q} c_{\downarrow,q} - \frac{1}{2} \delta_{Q,0}
	}_{D_{-Q}}
	\Big)
\, .
\end{gather*}

A perturbative expansion of the Green's function exploits (consider w.l.o.g. $c_{\omega} = c_{\uparrow, \omega}$) 
%
\begin{equation*}
\beta G(i\omega) = - \langle c_{\omega} \bar{c}_{\omega} \rangle
	= - \frac{\langle c_{\omega} \bar{c}_{\omega} e^{-S_{\text{int}}} \rangle_0}
	{\langle e^{-S_{\text{int}}} \rangle_0}
	= \beta G_{0, \omega} - \frac{1}{2} \langle 
	\big( c_{\omega} \bar{c}_{\omega} + \beta G_{0, \omega} \big)
	 S_{\text{int}}^2 \rangle_0 + \mathcal{O}(U^3)
\, .
\end{equation*}
%
Here, first order terms vanish due to Wick's theorem and the fact that without interaction, the resulting tight-binding model at zero chemical potential is half filled in the ground state,
%
\begin{equation*}
\sum_{\omega} G_{0, \omega} = \langle n_{\sigma} \rangle_0 = \frac{1}{2}
\quad
\Rightarrow
\quad
\langle C_Q \rangle_0 =  \big( \sum_{k} G_{0, \omega} - \frac{1}{2} \big) \delta_{Q,0} = 0
	= \langle D_Q \rangle_0
\, .
\end{equation*}
%
For the contribution to second order, note that only mixed terms survive,
%
\begin{equation*}
\langle D_{-Q_1} D_{-Q_2} \rangle_0
	= \sum_{q_1, q_2}
		\langle c_{\downarrow,q_2} \bar{c}_{\downarrow,q_1-Q_1} \rangle_0
		\langle c_{\downarrow,q_1} \bar{c}_{\downarrow,q_2-Q_2} \rangle_0
	= - \delta_{Q_2, -Q_1} \sum_{q} G_{0,q}G_{0,q+Q_1},
\quad
\text{and}
\end{equation*}
%
\begin{equation*}
\sum_{Q_1} \langle \big( c_{\omega} \bar{c}_{\omega}  +  \beta G_{0, \omega} \big) C_{Q_1} C_{-Q_1} \rangle_0 
	= 2 \sum_{k_1, k_2, Q_1}
		\langle c_{\omega} \bar{c}_{k_1+Q_1} \rangle_0
		\langle c_{k_2} \bar{c}_{\omega} \rangle_0
		\langle c_{k_1} \bar{c}_{k_2-Q_1} \rangle_0
	= -2 \beta G_{0,\omega}^2 \sum_k G_{0,k}
\, .
\end{equation*}

It follows that up to second order, the Green's function is given by
%
\begin{equation*}
G(i\omega) = G_{0, \omega} - U^2 G_{0,\omega}^2 
	\sum_k G_{0,k} \sum_{q} G_{0,q}G_{0,q-k+\omega}
	= G_{0, \omega} + G_{0,\omega}^2 \Sigma_{\omega}
\end{equation*}
%
where we defined the self energy $\Sigma$ in second order perturbation theory. It takes a simpler form in imaginary time space and remembering that we used an effective interaction, we summarize 
%
\begin{equation}
\Sigma(\tau) = - U^2 G_0(\tau)^2 G_0(-\tau)
\quad \quad
\text{with}
\quad 
\mu_{\text{eff}} = 0
\, .
\end{equation}


\section{On the spectral function}

As a measure of single-particle excitation, we consider the spectral function, which is obtained by analytic continuation from the Matsubara Green's function with the known principles:
%
\begin{equation}
\mathcal{A}(w) = - \text{Im} G(i\omega \rightarrow \omega + i0^+) / \pi,
\quad
\mathcal{A}(w) \geq 0,
\quad
\int_{-\infty}^{\infty} \mathcal{A}(w) \dif \omega = 1,
\quad
G(-i\omega) = G(i\omega)^*
\, .
\end{equation}
%
For a basic propagator of the type $G^{-1} \propto i\omega + a, a \in \mathbb{R}$, it is easily evaluated and yields
%
 \begin{equation}
 G(i\omega) = \frac{c}{i\omega+a} 
 \quad
 \Rightarrow
 \quad
\mathcal{A}(w) = \delta(w+a)
\quad
\Rightarrow
\quad
c = 1 
\, .
 \end{equation}














\begin{appendix}
\section{Implementation Details}
\subsubsection{Matsubara Frequencies and Fast Fourier Transform}
In order to solve the impurity model we have to perform several Fourier Transform.
As we consider electrons, the Green's function in imaginary time is antiperiodic by shifts of $\beta$, so we have to use fermionic Matsubara frequencies $ω_n:=\frac{π(2n+1)}{β}$.
The Fourier Transformations are given by:

\begin{align}
  G(i ω_n) &:= \int_0^β dτ G(τ) e^{i ω_n τ}\\
  G(τ) &= \frac{1}{β} \sum_{i ω_n} G(i ω_n) e^{-i ω_n τ}
\end{align}
For effient calculations we use the FFT-algorithm of the numpy package. Therefore we have to adapt our definitions to implementation of the numpy library. The numpy library calculates its Fourier Transform defined as:
\begin{equation}
  A_k = \mathrm{FFT}(a_m) =  \sum_{m=0}^{n-1} a_m \exp\left\{-2\pi i{mk \over n}\right\}
   \qquad k = 0,\ldots,n-1.
\end{equation}
Hence, we discretize the Matsubara Fourier transform
\begin{align}
  G(i ω_{-n}) &\approx \sum_{k=0}^{N-1} \Delta τ \, G(\Delta τ \, k) \exp{\left(i \frac{π (-2n+1)k}{N}\right)}\\
          &=\frac{\beta}{N} \sum_{k=o}^{N-1} \left( G(\Delta τ \, k)\exp{\left(i π \frac{k}{N}\right)}  \right)  \exp{\left(i \frac{-2 π n k}{N}\right)}\\
          &= \frac{\beta}{N} \mathrm{FFT}\left( G(\Delta τ \, k)\exp{\left(i π \frac{k}{N}\right)}\right)
\end{align}
where $\Delta τ = \frac{\beta}{N}$.
The same can be carried out for the inveres Fourier tranform.
\begin{equation}
  G(τ_k) = \frac{N}{β} e^{-i π \frac{k}{N}}\frac{1}{N}\sum_{ω_n}G(i ω_n) e^{-i 2π n k/N}
\end{equation}

\end{appendix}

\end{document}
