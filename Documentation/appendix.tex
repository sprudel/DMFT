\begin{appendix}
\section{Matsubara Frequencies and Fast Fourier Transform}
In order to solve the impurity model we have to perform several Fourier Transform.
As we consider electrons, the Green's function in imaginary time is antiperiodic by shifts of $\beta$, so we have to use fermionic Matsubara frequencies $ω_n:=\frac{π(2n+1)}{β}$.
The Fourier Transformations are given by (no implicit $\beta$):

\begin{equation}
  G(i ω_n) := \int_0^β dτ G(τ) e^{i ω_n τ} ,
  \quad
  G(τ) = \frac{1}{β} \sum_{i ω_n} G(i ω_n) e^{-i ω_n τ}
\end{equation}
%
For effient calculations we use the FFT-algorithm of the numpy package. Therefore we have to adapt our definitions to the implementation of the numpy library. The numpy library calculates its Fourier Transform by:
\begin{equation}
  A_k = \mathrm{FFT}(a_m) =  \sum_{m=0}^{n-1} a_m \exp\left\{-2\pi i{mk \over n}\right\}
   \qquad k = 0,\ldots,n-1.
\end{equation}
Hence, we discretize the Matsubara Fourier transformation
\begin{align}
  G(i ω_{-n}) &\approx \sum_{k=0}^{N-1} \Delta τ \, G(\Delta τ \, k) \exp{\left(i \frac{π (-2n+1)k}{N}\right)}\\
          &=\frac{\beta}{N} \sum_{k=o}^{N-1} \left( G(\Delta τ \, k)\exp{\left(i π \frac{k}{N}\right)}  \right)  \exp{\left(i \frac{-2 π n k}{N}\right)}\\
	  &= \frac{\beta}{N} \mathrm{FFT}\left( G(\Delta τ \, k)\exp{\left(i π \frac{k}{N}\right)}\right), \label{eq:MFFT}
\end{align}
where $\Delta τ = \frac{\beta}{N}$.
The same can be carried out for the inverse Fourier Tranformations.
\begin{align}
	G(\Delta τ k) &= \frac{N}{β} e^{-i π \frac{k}{N}}\frac{1}{N}\sum_{ω_n}G(i ω_{-n}) e^{i 2π n k/N}\\
	&= \frac{N}{β} e^{-i π \frac{k}{N}}\mathrm{IFFT}(G(iω_{-n})) \label{eq:IMFFT}
\end{align}
\begin{figure}[h]
	\centering
	\includegraphics[width=\textwidth]{Matsubara_Fourier_fig}
	\caption{Comparison of the different discretized Fourier Transformations. The improved version, by manually removing the $\frac{1}{i \omega}$ factor, approximates the exact transformation significantly better. }
	\label{fig:fourier_traf}
\end{figure}

Unfortunately the ``naive'' implementations \eqref{eq:MFFT} and \eqref{eq:IMFFT} cause numerical problems, since according to \eqref{G_largefreq} Green's function only decay as $1/ i\omega_n$ in frequency space. As the frequency sum is cut off by the finite number of points used, one strongly increase the accuracy by manually transforming the $1/ i\omega_n$ part. With contour integration, it is commonly shown that ($\tau \neq 0$)
%
\begin{gather}
G(i\omega_n) = \frac{1}{i\omega + a}
\quad \Leftrightarrow \quad
G(\tau) = \Theta(\tau) \frac{- e^{a \tau}}{e^{\beta a}+1} + \Theta(-\tau) \frac{e^{a \tau}}{e^{-\beta a}+1} 
\\
	G(i ω_n) =\frac{1}{i ω_n} \quad ⇔ \quad G(τ)=-\frac{1}{2} + \Theta(-\tau)
	\label{eq:ff_pair}
\, .
\end{gather}
Consequently the improved version of our Fourier transformation is given by subtracting and adding the relevant terms before and after the transformation.
\begin{align}
	G(i ω_{-n})&= \frac{1}{i ω_{-n}}+\frac{\beta}{N} \mathrm{FFT}\left( \left(G(\Delta τ \, k)+\frac{1}{2}\right)\exp{\left(i π \frac{k}{N}\right)}\right)\\
	G(\Delta τ k)&= -\frac{1}{2}+\frac{N}{β} e^{-i π \frac{k}{N}}\mathrm{IFFT}\left(G(iω_{-n})-\frac{1}{i ω_{-n}}\right)
	\label{eq:improved_fft}
\end{align}
The improvement can be seen in \figref{fig:fourier_traf}, where we compare the exact Fourier transformation of $G(i ω)=\frac{1}{iω+a}$ to our discretized versions. The naiv version shows significant deviations to the analytic solution, whereas our improved version approximates the exact one very well.  

\section{Analytic continuation}
For the anytic continuation of the Matsubara Green's function we have to estimate the functional dependence of $G(iω_n)$. We achived this by using the so called Pade approximation, which fits a rational function to our given discrete values. A efficient way to calculalte the Pade approximation can be found in \todo{ref paper or repeat everything here?}. As the rational function is continous and the Matsubara Green's function exhibits a jump near zero frequency, we can either use data points at positive frequencies or negative frequencies. We assume that the Green's function is analytic in the upper and lower half plane of the complex plane \todo{is this true??????}. Therefore, when calculate the retarded Green's function, which lies slightly above the real axis, we have to use the fit corresponding to the positive frequencies on the imaginary axis. \todo{maybe a plot here?}. 


\end{appendix}
